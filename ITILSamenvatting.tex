\documentclass{article}

%meta data
\date{\today}
\title{ITIL Samenvatting}
\author{Wannes De creane, Michiel Schoofs}

%pacakage imports
\usepackage[dutch]{babel}
\usepackage[backend=biber,style=apa,autocite=inline]{biblatex}

%bib reference
\addbibresource{Bibliografie.bib}

%Custom commando's
\newcommand{\boldit}[1]{\emph{\textbf{#1}}} 
\newcommand{\customref}[1]{\underline{\ref{#1}: \nameref{#1}}}

\begin{document}
	\maketitle
	\section{Inleiding}
	ITIL staat voor Information Technology Infrastructure Library, Dit is een methodiek om aan procesmatig werken te doen binnen in IT. Meerbepaald een praktische “no nonsense approach”, zoals beschreven volgens ITIL®: the basics van \cite{Cater-Steel2006}, voor identificatie, planning, levering en support van IT services voor bedrijven.\\
	
	\par
	\noindent
	ITIL is aldus een leidraad om aan IT Service Management ( ITSM ) te doen, maar wat is dit nu juist? Een service voorziet waarde voor klanten, services die direct door klanten kunnen gebruikt worden noemt men bijgevolg business services een voorbeeld hiervan is bijvoorbeeld “Payroll”, dit is een IT service die gebruikt wordt om informatie bij te houden, compensatie te berekenen en cheques te genereren.\\
	
	\par
	\noindent
	Vaak ziet men de verschillende services van IT en business als een volledig andere wereld. Men gaat dan ook vaak gaan micromanagen en verliest het grote plaatje uit het oog. ITIL suggereert echter een meer holistische en consequente benadering van het ITSM process.\\
	
	\par
	\noindent
	Om de samenhang tussen nieuwe IT services en het bedrijf zonder problemen te laten werken zijn er dus verschillende processen nodig.
	
	
	\printbibliography
\end{document}